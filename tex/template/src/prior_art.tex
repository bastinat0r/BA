\section{Finken Robots}

The Finken project aims to create a swarm of autonomously flying quadrocopters to research swarm intelligence beheaviour on robots.
Many algorithems in swarm intelligence are based on distance-values. \todo{source}
For this reason it is reasonable to search for a sensor that is capable to measure distances and integrate it into the Finken robots. 

The Finken robots are already existing and of course it is necessary to know wich kind of hardware on the robots could interfere with the ranging sensors that shall be integrated into those robots.

\begin{itemize}
	\item[IMU] Inertial Measurement Unit with accelerometer, magnetometer and barometer
	\item[Sonar Sensors] Sonar sensors to measure distances of the nearest object in four directions (front, back, left, right)
	\item[IR-Sensor] Sensor to measure distance to ground with high frequency
	\item[Optical Flow] Optical flow sensor, that can be integrated to measure x-y-velocity over ground
	\item[Motors] Four brushless motors that may cause RF-interfercene and noise
	\item[Telemetry] BTLE-/Zigbee modules to exchange data with the ground station
	\item[RC-Control] 2.4GHz based Radio Control to manually control the robots
	\item[Power-Supply] Lithium polymer batteries with nominally 6.6V \todo{fink3?} output voltage that is converted to 5V and 3.3V by the power distribution hardware
	\item[Payload] The overall weight of the copter in the current configuration is about \todo{weight}g with about \todo{payload}g headroom for additional equipment
	\item[Size] The copter has a rotor to rotor distance of 10cm, and a sensor tower that is about 4cm by 4cm wide to use the existing mounting holes would be favourable
\end{itemize}

\section{Evaluation of Existing Ranging Solutions}
There are some technologies that can be used for ranging, however the usual application for most of those technologies in research is positioning.
For that reason it is interesting to search for positioning applications that use range measurements, however many of those positioning technologies are base on other principles than multilateration\footnote{The usual methods for positioning are: \emph{multilateral}—which is what we are interested in because only ranging measurments are used, \emph{multiangular}—which is no use  to us, because angle measurements are used and by \emph{orientating in a map} with different factors like beacon-positions–which is also no use to us.}.

The usual technologies used for ranging are based on time of flight measurments, signal strength, optical tracking, and phase diffence measurments in signals.

\subsection{Indor Time of Flight}
The obvious approach for replacing the GPS signal that is aviable outdoors is to use a simmilar approach indoors.
http://robotics.eecs.berkeley.edu/~pister/290Q/Papers/Location/Lanzisera\%20RF\%20TOF\%20WISES06.pdf \todo{quellify} states, that an accuracy of $2.6m_{RMS}$ was achieved indoors.
With an operating area only $2m$ wide this approach is not suited for our robots.
However this research is focused on using cheap sensor-nodes.
 \todo{find commercial solutions with better accuracy}

http://www.researchgate.net/profile/Bardia\_Alavi/publication/224315086\_Measurement\_and\_Modeling\_of\_Ultrawideband\_TOA-Based\_Ranging\_in\_Indoor\_Multipath\_Environments/links/0912f50b396c340971000000.pdf \todo{quellify}

Another approach to provide an indoor GPS-like solution is iGPS. http://www.nikonmetrology.com/de\_EU/Produkte/Grossvolumige-Messaufgaben/iGPS/iGPS \todo{quellify} however is not ranging-based but uses angulation as underlying technology and is therefore useless to us.
  * IGPS http://porto.polito.it/2438175/2/IJAMT\_iGPS\_and\_LT.pdf
\subsection{Cricket / Active Bat}
A very clever approach to ranging is used by ranging solutions like cricket and active bat\todo{Quelle, Quelle}. 
RF-Signals travel at the speed of light and therefore you need to be able to measure very short timings in time of flight scenarios.
Sound however travels at a speed much slower than RF.
Cricket and Active Bat use this to measure the time difference an RF-signal and an ultrasound pulse need to travel from transmitter to reciever to calculate the range between two sensor nodes. \todo{thunderstorm and ligthning very very frightning}

\todo{accuracy / price, moving objects, medium access (number of nodes)}


There are two big problems with this approach that stem from the current setup of the Finken-Robots.
The Finken Robots use ultrasound sensors to measure the distances to nearby objects.
Those technologies would interfere with the ultrasound sensors already used and a replacement would be needed.

Another problem is the noise created by motors and propellers.
The sound made by the quadrocopters is not ending in the hearable spectrum but also extends to the ultrasound range.
\todo{measure noise, PWM-frequency of speedcontrollers}

\subsection{RSSI-based ranging}

A property that can be used to do RF-based ranging is signal strength.
The further the source of the signal is away the weaker the signal gets.
RSSI-based ranging is done for serveral different technologies: Bluetooth\todo{quelle}, WLAN\todo{quelle}, RFID\todo{quelle} –
There are even approaches using maps created of different RSSI-ranging sources. http://www.gnss.com.au/JoGPS/v9n2/JoGPS\_v9n2p122-130.pdf \todo{quellify}

The main factor that rules out RSSI-based ranging is that radio-waves are not propageted equally in every direction. \todo{typical propagation pattern picture}
Antenna-orientation might have a much bigger impact on signal strength than distance.
Additionally radio waves might be weakend when travelling through the Finken-Robots and by doing so passing wires and electronic components.

\subsection{External Tracking}




\subsection{Atmel RTB, Dresden Elektronik, Meterionic}

