\section{Motivation}

The FINken project aims to create a swarm of autonomously flying quadcopters to research swarm intelligence beheaviour on robots.
Many algorithms in swarm intelligence are based on distance values. \todo{source}
On this occasion it is important to find a sensor that is capable to measure distances and to integrate it into the FINken robots.
An obvious choice for this distance is measuring the spacial distance in between the robots\footnote{
The mathematical distance-property might not actually need to be something as close to a real \emph{physikalische Größe}, but it is great to be able to measure and watch the robots interacting with their physical environment.}.
\todo{wording}

\section{The FINken Robot Platform}

Goal of the FINken project is to do research on how swarm beheaviour can be executed by flying robots.
Therefore suitable robots need to be constructed.
The robots need to fly in an stable manner on their own and be capable of interacting without disrupting the operation of the other swarm members.
Those robots shall perform given tasks defined to encurage swarm based interaction, their beheaviour can be evaluated and compared to the theoretical models developed by swarm intelligence research.

\subsection{Robot Description}
\todo{ein bild sagt mehr als tausend worte}
The robots are propelled by four rotors that are directly attached to a brushless motor.
In combination the motors can be controlled to change the direction of the thrustvector (pitch and roll), to change the overall amount of lift generated (thrust) and to change the orientation of the airframe (yaw).
The airframe houses all the actuators, processors and batteries needed for flight.
Additionally it carries a multitude of sensors used for operating autonomsly.

The robots are capable of highly dynamic flight maneuvers–the robots have enough acceleration to leave the operating environment in any possible direction in under one second.
This is mainly because the robots need lots of payload capacity to carry different sensors and computing power, with enough headroom to make changes in the future.
However the capability for dynamic beheaviour is not usefull for our research.
Even if the high power of the motors is not utilized stability \todo{find better word} is a problem.
If the copter is angled by only $6\degree$ and it's height is kept stable it is accelerating at about \SI{1}{\metre\per\square\second} if the height stays the same during that time.
This means the copter reaches a velocity of over \SI{2}{\metre\per\second} when travelling through the arena at this pretty small angle.

\subsection{Environment}
Creating a swarm of flying robots is a rather difficult task.
The environment for the FINken robots is created to protect the robots from mechanical damage and to function well with the sensors the robot uses.

The FINken robots fly in an area of \SI{2}{\metre} by \SI{3}{\metre} that can be expanded to about \SI{3}{\metre} by \SI{4}{\metre}.
The flight area is enclosed by netting and ultrasound-reflecting foil those barriers act the same way a wall would without damaging the robots if they fail to elude them.
Usually the altitude of operation is between \SI{0.5}{\metre} to \SI{1}{\metre}.
To prevent damage when the quadcopters crash the floor is coverd with mats that work well with ultrasound and infrared sensors.
It is possible to create virtual environmental factors by using a projector and an rgb-sensor mounted on top of the robots.
This virtual environment can be used assign a certain task to the robot, e.g. finding the brightest spot.
\todo{foto}


\subsection{Sensor Concept}
The sensors used by the FINken robots serve two purposes: To enable the robots to fly autonomsly and to interact with other robots and the environment.

\subsubsection{Autonomous Flying}
To form a swarm the robots need to function as single individuals first-that means not crashing into walls, ceiling, floor or other robots.
Of course the sensors needed for autonoms flight must not be disturbed by anything else. 

The flight dynamics of the FINken robots shall be as simple as possible.
The robots shall be able to fly at a given height and navigate in x-y-direction to perform their given tasks.
Highly dynamic maneuvers are possible for the airframe, but are not realizable with the sensor data we currently get. \todo{ref to christophs ba}

For flying autonomously two major problems have to be solved: Height control and navigation.
For controlling the height of the copters either the ultrasound distance measurement of the optical flow sensors are used, alternativly the height is measured by an infrared sensor that must be equipped when the optical flow sensors are not.

Navigating the x-y-plane is a much harder task.
Because sensors are not perfect the copter will drift in any given direction and precise speed and position measurements are hard to get by.
As a consequence implementing a position hold mode is quite a challenge. \todo{wording}
Assuming that a stable position can not be kept there is still something that can be done to fly for longer time periods.
With an ultrasound distance sensor in each direction the robot can sense its surroundings and avoid any obstacles in its vicinity.
An essential precondition for navigating around obstacles that way is a certian level of stability and avoiding highly dynamic maneuvers.

\subsubsection{Interaction}

The FINken robots shall be able to do more than simply not crashing.
Of course sensor values needed for autonomous flight can also be used to interact with the robots environment, but there are also senors that are exclusively used for interaction.

To interact with an environment a solution is necessary that works well in the laboratry.
The environment of the robots shall be changed without changing the physical layout of the lab.
An environmental factor that can be easily changed is the lighting of the flying area.
In the swarmlab a projector can be used to affect the quadcopters.
To measure the current lighting situation an rgb color sensor is mounted on top of the copter.

The range-sensor is also mainly used for interaction.
Ranging values enable the FINken robots to implement attraction-repulsion-beheaviour.

\subsection{Hardware Description}
There are different ranging technologies that might be used in a FINken quadcopter.
However there are different components that can interfere with the new sensor that shall be integrated e.g. by disturbing the measurments made by the new sensor.

%The FINken 3 robots are quadcopters controlled by the Lisa\/MX 2.1 autopilot-board\cite{_lisa/mx_????} using the opensource autopilot firmware Paparazzi\cite{_paparazziuav_????}.
\begin{table}[H]
	\begin{tabularx}{\columnwidth}{l | X}
	Part & Description \\ \hline
	Sonar Sensors & Sonar sensors to measure distances of the nearest object in four directions (front, back, left, right) \\
	Motors & Four brushless motors that may cause RF-interfercene and noise \\
	Telemetry & BTLE-/Zigbee modules to exchange data with the ground station \\
	RC-Control &  2.4GHz based Radio Control to manually control the robots \\
	Power-Supply & Lithium polymer batteries with nominally \SI{11.1}{\volt} output voltage that is converted to 5V and 3.3V by the power distribution hardware \\
	Payload & The overall weight of the copter in the current configuration is about \\ %\todo{weight}g with about \todo{payload}g headroom for additional equipment \\
	Size & The copter has a rotor to rotor distance of 10cm, and a sensor tower that is about 4cm by 4cm wide to use the existing mounting holes would be favourable \\
	\end{tabularx}
	\caption{Hardware Components of the FIKen 3 robots}
\end{table}

\section{Evaluation of Existing Ranging Solutions}
\todo{mehr fokus auf andere copter projekte}
There are some technologies that can be used for ranging, however the usual application for most of those technologies in research is positioning.
For that reason it is interesting to search for positioning applications that use range measurements, however many of those positioning technologies are base on other principles than multilateration\footnote{The usual methods for positioning are: \emph{multilateral}—which is what we are interested in because only ranging measurments are used, \emph{multiangular}—which is no use  to us, because angle measurements are used and by \emph{orientating in a map} with different factors like beacon-positions–which is also no use to us.}.
\cite{_multilateration_2015}

The usual technologies used for ranging are based on time of flight measurments, signal strength, optical tracking, and phase diffence measurments in signals.

\subsection{Indor Time of Flight}
The obvious approach for replacing the GPS signal is to use a simmilar approach indoors.

The problem is that very short timespans have to be measured accuratly, because radio waves are so damn fast. \todo{wording}
\cite{lanzisera2006} states, that standard errors of \SI{2.6}{\metre_{RMS}} and \SI{1.8}{\metre_{RMS}} was measured in different indoor scenarios.
With an operating area only \SI{3}{\metre} wide this approach is not suited for our robots.


% \cite{uwb_decawave} claims that an accuracy of \SI{10}{\centi\metre} can be achieved and the nodes are aviable for purchase at reasonable prices. \todo{begründen warum wir das nicht benutzen} \todo{wiederspruch mit den 10cm, more like 50cm}
A project that looks more promising is decaWave.
According to \cite{uwb_localisation_copter} the measurement error is generally under \SI{1}{\metre} and with filtering can be brought to below \SI{15}{\centi\metre}.
The proposed method of filtering needs 27 measurements to compute one range value.\todo{wording}
However the decaWave project was unkown to us when we started looking for ranging possibilities so \todo{blabla}

\subsection{Cricket / Active Bat}
A very clever approach to ranging is used by ranging solutions like cricket\cite{cricket_01} and active bat\cite{active_bat}. 
RF-Signals travel at the speed of light and therefore you need to be able to measure very short timings in time of flight scenarios.
Sound however travels at a speed much slower than radio waves.
The same principle can be applied when you measuring the distance of a lightning strike by measuring the time between thunder and lightning.
Cricket and Active Bat also measure the time difference an RF-signal and an ultrasound pulse need to travel from transmitter to reciever to calculate the range between two sensor nodes.

A bad side effect of using sound as medium to the ranging method is that there is an upper bound to the update frequency for all nodes sharing the medium (i.e. close enough to sense each other). \cite{active_bat} claims that one ranging measurement can be done in a \SI{20}{\milli\second} slot.
That means we can have 50 range updates per second. Assuming we have a swarm of five robots that form a fully connected graph we would need at least ten range measurements to get all swarm distances.
So the upper boundary for ranging update frequency in a swarm of five robots is \SI{5}{\hertz}.
Considering that this is the upper limit this is a solid disadvantage of this method.
Furthermore the FINken robots currently use sonar based distance sensor to measure the distance to the nearest object in some direction so either the update frequency would be diminished much further or the sonar sensors needed to be replaced
\todo{accuracy / price, moving objects}

Ultrasonic ranging on quadcopters is done by \cite{ultrasonic_erlangen}

Another problem is the noise created by motors and propellers.
The sound made by the quadcopters is not ending in the hearable spectrum but also extends to the ultrasound range.
In conclusion ultrasound based ranging is a very neat approach that could be integrated into the FINken robots, but the current setup that already includes sonar sensors make this approach impractical.

\subsection{RSSI-based ranging}

A property that can be used to do RF-based ranging is signal strength.
The further the source of the signal is away the weaker the signal gets.
RSSI-based ranging is done for serveral different technologies: Bluetooth\todo{quelle}, WLAN\todo{quelle}, RFID\todo{quelle} –
There are even approaches using maps created of different RSSI-ranging sources. http://www.gnss.com.au/JoGPS/v9n2/JoGPS\_v9n2p122-130.pdf \todo{quellify}

The main factor that rules out RSSI-based ranging is that radio-waves are not propageted equally in every direction. \todo{typical propagation pattern picture}
Antenna-orientation might have a much bigger impact on signal strength than distance.
Additionally radio waves might be weakend when travelling through the FINken robots and by doing so passing wires and electronic components.

\subsection{External Tracking}

Most projects use external tracking to measure the position and orientation of the quadcopters. \todo{refs}
The most common optical tracking systems are very costly in comparison to the other ranging methods described here.
This price is justified by the superior performance of this method.

This means tracking accuracy as well as the possibility to measure orientation.
All that at a high update frequency.
\todo{research performance statistics for ranging solutions}

\todo{ETH, tracking, sensorik foo}

A huge drawback to this method is that many components are used that need to be integrated into the environment and cannot be carried by the robots themselves.
For swarm robotics this is not an ideal solution as using external tracking would mean communicating with some kind of centralized tracking interface-destroying the scalability and the priciple of local interaction leading to global beheaviour.
\todo{price}
\subsection{Atmel RTB, Dresden Elektronik, Meterionic}

Another thing that can be measured phase shift.
This is a principle that is used by some of the ultrasonic methods.
However the phase difference can also be measured in radio waves.
This is utilized by the ranging hardware from the Atmel Ranging Toolbox.
Multiple frequencys are used to measure a phase difference.
Because the wave length changes with different frequencys you can take all of the measured phase differences and compute a distance.
\todo{quelle}
Similar hardware using the same software stack is also sold by \emph{Dresden Elektronik} and \emph{Meterionic}.

Using phase differences in RF discards the medium access problems of ultrasonic methods as well as the wave propagation problems of RSSI-based methods.
Therefore it seems like a feasible approach for the FINken robots.

