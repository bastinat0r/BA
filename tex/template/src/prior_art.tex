\section{The FINken Robot Platform}


The FINken project aims to create a swarm of autonomously flying quadcopters to research swarm intelligence beheaviour on robots.
Many algorithems in swarm intelligence are based on distance-values. \todo{source}
On this occasion it is important to find a sensor that is capable to measure distances and to integrate it into the FINken robots. 

\subsection{Environment}
Creating a swarm of flying robots is a rather difficult task.
Therefore the environment for the FINken robots is created to protect the robots from mechanical damage and to function well with the sensors the robot uses.

The FINken robots fly in an enclosed area of 2m by 3m that can be expanded to about 4m by 3m.
The flight area is enclosed by netting and ultrasound-reflecting foil.
Usually the altitude of operation is between 50cm to 1m.
To prevent damage when the quadcopters crash the floor is coverd with mats that work well with ultrasound and infrared sensors.
It is possible to create virtual environmental factors by using a projector and an rgb-sensor mounted on top of the robots.
This virtual environment can be used assign a certain task to the robot, e.g. finding the brightest spot.
\todo{foto}

\subsection{Actuators and Dynamic} \todo{put this subchapter into introduction}
The FINken Robot is a quadcopter.
Like most quadcopters the FINken Robots are propelled by four rotors that are directly attached to a brushless motor.
In combination the motors can be controlled to change the direction of the thrustvector (pitch and roll), to change the overall amount of lift generated (thrust) and to change the orientation of the airframe (yaw).
The speed of each motor is controlled by an the Lisa\/MX 2.1 autopilot-board\cite{_lisa/mx_????} using the opensource autopilot firmware Paparazzi\cite{_paparazziuav_????}.

The robots are highly dynamic-the robots have enough acceleration to leave the operating environment in any possible direction.
This is mainly because the robots need lots of payload capacity to carry different sensors and computing power, with enough headroom to make changes in the future.


\subsection{Sensor Concept}
The sensors used by the FINken robots are usually used for two purposes: To enable the robots to fly autonomsly and to interact with other robots and the environment.

\subsubsection{Autonomous Flying}
To form a swarm the robots need to function as single individuals first-that means not crashing into walls, ceiling, floor or other robots.
Of course the sensors needed for autonoms flight must not be disturbed by anything else. 

\todo{height control}
\todo{wall avoid}

\subsubsection{Interaction}

The FINken robots shall be able to do more than simply not crashing.
Of course sensor values needed for autonomous flight can also be used to interact with the robots environment, but there are also senors that are exclusively used for interaction.

To interact with an environment a solution is necessary that works well in the laboratry.
The environment of the robots shall be changed without changing the physical layout of the lab.
An environmental factor that can be easily changed is the lighting of the flying area.
In the swarmlab a projector can be used to affect the quadcopters.
To measure the current lighting situation an rgb color sensor is mounted on top of the copter.

The range-sensor is also mainly used for interaction.
Ranging values enable the FINken robots to implement attraction-repulsion-beheaviour.

\subsection{Hardware interfering with ranging}
There are different ranging technologies that might be used in a FINken quadcopter.
However there are different components that can interfere with the new sensor that shall be integrated e.g. by disturbing the measurments made by the new sensor.

\begin{itemize}
	\item[Sonar Sensors] Sonar sensors to measure distances of the nearest object in four directions (front, back, left, right)
	\item[Motors] Four brushless motors that may cause RF-interfercene and noise
	\item[Telemetry] BTLE-/Zigbee modules to exchange data with the ground station
	\item[RC-Control] 2.4GHz based Radio Control to manually control the robots
	\item[Power-Supply] Lithium polymer batteries with nominally 6.6V \todo{fink3?} output voltage that is converted to 5V and 3.3V by the power distribution hardware
	\item[Payload] The overall weight of the copter in the current configuration is about \todo{weight}g with about \todo{payload}g headroom for additional equipment
	\item[Size] The copter has a rotor to rotor distance of 10cm, and a sensor tower that is about 4cm by 4cm wide to use the existing mounting holes would be favourable
\end{itemize}

\section{Evaluation of Existing Ranging Solutions}
\todo{mehr fokus auf andere copter projekte}
There are some technologies that can be used for ranging, however the usual application for most of those technologies in research is positioning.
For that reason it is interesting to search for positioning applications that use range measurements, however many of those positioning technologies are base on other principles than multilateration\footnote{The usual methods for positioning are: \emph{multilateral}—which is what we are interested in because only ranging measurments are used, \emph{multiangular}—which is no use  to us, because angle measurements are used and by \emph{orientating in a map} with different factors like beacon-positions–which is also no use to us.}.
\cite{_multilateration_2015}

The usual technologies used for ranging are based on time of flight measurments, signal strength, optical tracking, and phase diffence measurments in signals.

\subsection{Indor Time of Flight}
The obvious approach for replacing the GPS signal that is aviable outdoors is to use a simmilar approach indoors.
http://robotics.eecs.berkeley.edu/~pister/290Q/Papers/Location/Lanzisera\%20RF\%20TOF\%20WISES06.pdf \todo{quellify} states, that an accuracy of $2.6m_{RMS}$ was achieved indoors.
With an operating area only $2m$ wide this approach is not suited for our robots.
However this research is focused on using cheap sensor-nodes.
 \todo{find commercial solutions with better accuracy}

http://www.researchgate.net/profile/Bardia\_Alavi/publication/224315086\_Measurement\_and\_Modeling\_of\_Ultrawideband\_TOA-Based\_Ranging\_in\_Indoor\_Multipath\_Environments/links/0912f50b396c340971000000.pdf \todo{quellify}

Another approach to provide an indoor GPS-like solution is iGPS. http://www.nikonmetrology.com/de\_EU/Produkte/Grossvolumige-Messaufgaben/iGPS/iGPS \todo{quellify} however is not ranging-based but uses angulation as underlying technology and is therefore useless to us.
  * IGPS http://porto.polito.it/2438175/2/IJAMT\_iGPS\_and\_LT.pdf

\todo{deka-wave}
\subsection{Cricket / Active Bat}
A very clever approach to ranging is used by ranging solutions like cricket and active bat\todo{Quelle, Quelle}. 
RF-Signals travel at the speed of light and therefore you need to be able to measure very short timings in time of flight scenarios.
Sound however travels at a speed much slower than RF.
Cricket and Active Bat use this to measure the time difference an RF-signal and an ultrasound pulse need to travel from transmitter to reciever to calculate the range between two sensor nodes. \todo{thunderstorm and ligthning very very frightning}

\todo{accuracy / price, moving objects, medium access (number of nodes)}


There are two big problems with this approach that stem from the current setup of the FINken-Robots.
The FINken Robots use ultrasound sensors to measure the distances to nearby objects.
Those technologies would interfere with the ultrasound sensors already used and a replacement would be needed.

Another problem is the noise created by motors and propellers.
The sound made by the quadcopters is not ending in the hearable spectrum but also extends to the ultrasound range.
\todo{measure noise, PWM-frequency of speedcontrollers}

\subsection{RSSI-based ranging}

A property that can be used to do RF-based ranging is signal strength.
The further the source of the signal is away the weaker the signal gets.
RSSI-based ranging is done for serveral different technologies: Bluetooth\todo{quelle}, WLAN\todo{quelle}, RFID\todo{quelle} –
There are even approaches using maps created of different RSSI-ranging sources. http://www.gnss.com.au/JoGPS/v9n2/JoGPS\_v9n2p122-130.pdf \todo{quellify}

The main factor that rules out RSSI-based ranging is that radio-waves are not propageted equally in every direction. \todo{typical propagation pattern picture}
Antenna-orientation might have a much bigger impact on signal strength than distance.
Additionally radio waves might be weakend when travelling through the FINken robots and by doing so passing wires and electronic components.

\subsection{External Tracking}




\subsection{Atmel RTB, Dresden Elektronik, Meterionic}

