
The basic idea for the firmware of the nodes is to provide an I2C interface to initiate measurements and read values.
Additionally the settings of the modules can be changed by the same interface.

\section{Description of the Ranging Process}

While the ranging itself is treated as a black box in this thesis it is still needed to konw how a measurement is executed.
For ranging there are three functions a node can fullfill.
The \emph{reflector} is the target node of the ranging.
The \emph{initiator} is the node sending the original signal and measuring the actual phase difference of the reflected signal.
The measured range will be the distance between initator and refletor node.
The third node type is only needed for remote ranging.
The \emph{coordinater} node is used to trigger a range reading between two other ranging nodes.

Each of the ranging nodes is able to act as reflector, coordinator and reflector without changing the firmware.
If the nodes are configured correctly the nodes communicate about the pending measurement on a seperated RF-channel.

The measuring a range value works as follows:
\begin{enumerate}
	\item set initiator and refletor node
	\item start measurement
	\item wait for measurement to finish
	\item fetch the result
\end{enumerate}

As result of the ranging process two values are put out: Range and DQF.
The range value is the estimated range in cm.
The DQF is an additional parameter tells the user how accutrate the range value is assumed to be.

\section{Communication Interface}


\subsection{I2C Registers}
The sensor has different functions available as I2C-registers.
The master device writes one byte to the register followed by arguments for the different functions.
Either the sensor node answers with an acknowledge byte or the data answering the request.
The registers are listed in \autoref{i2ccommands}.
In normal operation the master device will set reflector and initiator address, initiate the ranging and read the resulting range value afterwards.
This is achieved with the START\_RANGING and READ\_LAST\_RANGING commands.

Additionally some basic configuration can be might via the I2C interface.
The I2C-address is the address of the I2C-device, that needs to be changed to avoid address collisions i.e. when multiple ranging nodes are used with one master device.
The short address is the address describing the ranging node.
It is independent from the I2C address in order to allow multiple ranging nodes on one I2C master device as well as using ranging devices with equal addresses on multiple devices.

The FINken robots will certainly share one I2C address for the ranging nodes and use the aircraft-ID of the robot as short address.
As the aircraft-ID of the FINken is only one byte the higher byte of the ranging short addresses might be used to distinguish between other robots in the swarm and nodes in the environment.

Initiator and reflector address refer to the short addresses of the nodes on both ends of one measurement.
The initiator is the ranging node starting the ranging process, the reflector its target.
Because the nodes are capable of remote ranging the initiator might be a different node than the one connected via I2C.
In particular this means that remote range readings can be taken without the need of additional communication.

\begin{table}
	
	\begin{tabularx}{\textwidth}{l | l | X}
	Byte & Name & Description \\ \hline
	0x0  & ECHO & return payload byte\\
	0x1  & START\_RANGING & trigger range measurement\\
	0x3  & START\_REMOTE\_RANGING & trigger remote measurement\\
	0x2  & READ\_LAST\_RANGING & read measured distance\\
	0xFE & SET\_I2C\_ADDRESS & set new I2C address\\
	0xFD & SET\_SHORT\_ADDRESS & set new ranging short address\\
	0xFC & SET\_REFLECTOR\_ADDRESS & set reflector address\\
	0xFB & SET\_INITIATOR\_ADDRESS & set initiator address\\
	0xED & GET\_SHORT\_ADDRESS & get ranging address\\
	0xEC & GET\_REFLECTOR\_ADDRESS & get reflector address\\
	0xEB & GET\_INITIATOR\_ADDRESS & get initiator address\\
	0xFF & CLEAR\_BUFFER & clear I2C write buffer\\
	0xCA & SET\_FREQ\_START & set lower ranging frequency\\
	0xCB & SET\_FREQ\_STEP & set ranging frequency spacing\\
	0xCC & SET\_FREQ\_STOP & set upper frequency\\
	0xCD & SET\_DIVERSITY & turn on/off antenna diversity\\
	\end{tabularx}

	\caption[Implemented I2C-Commands and Description]{Implemented I2C-Commands and Description.}
	\label{i2ccommands}
	
\end{table}

\subsection{Datafields in the Ranging Result}
\autoref{rangefields} describes how the structure for transmitting range values is organized.
The reason why so many values are included into the range measurement is that the master device is most propably needs to do filtering based on status and dqf-values. The addresses of the nodes are included to match measurements in case one of the packets is lost or a new measurement is made before the old value is read.

The data type for the range values is changed, not to block the I2C-device unnecessarily the data type of the range reading is changed.
Instead of the original 32-Bit value only a 16-Bit value is used, as distances up to more than \SI{60}{\metre} are more than plenty in our application.

\begin{table}
	
	\begin{tabularx}{\columnwidth}{l | l | X}
	Type & Name & Description \\ \hline
		uint8\_t  & status       & status of the range measurement \\
		uint8\_t  & dqf          & quality of the range reading \\
		uint16\_t & distance     & measured distance \\
		uint16\_t & short\_addr1 & initiator address \\
		uint16\_t & short\_addr2 & reflector address \\
	\end{tabularx}

	\caption{Fields included in one range measurement}
	\label{rangefields}
	
\end{table}



%\todo{"TWI", "Phillips-I2C", "..."}

%\section{Paparazzi Module for Ranging}
%
%\todo{Treiber für  Ranging}
%\todo{(optional)Treiber für Pseudo GPS}
%
%Addressing is done using the unique aircraft id of the indivdual robots.
%Because the short addresses for the ranging nodes are 16 bit long and the aircraft ids are 8 bit values it is feasable to use a fixed prefix together with the ac-id as node addresses.

\section{Python Scripts}

For testing the sensor nodes and collecting sample data a raspberryPi minicomputer was set up as an I2C master device.
The scripting language python was used to implement all the functions the I2C interface of the ranging nodes provide.

\emph{i2cranging.py} contains functions for the master side of I2C communication. Those can either be used from the python PEPL or by other scripts.
\emph{poll\_range.py} contains a convenient method to take continous range readings from the unix shell and is mainly used to generate csv-files with ranging values.
Those csv-files have been used for evaluating the ranging nodes.

Gathering data with those scripts may not only prove usefull for this work.
It might be an efficient approach to develop and evaluate algorithms for filtering and position estimation using higher level concepts.
Implementing only those algorithms on the embedded devices that prove to be usefull.



