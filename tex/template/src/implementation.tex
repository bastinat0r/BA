
\section{Hardware}

\subsection{Ranging Hardware}

\subsection{Autopilot}

\section{Interconnect}
There are diffrent solutions to connecting the ranging board to the Paparazzi autopilot.

\subsection{Pulse width modulation / Analog value}
Using a single GPIO pin or analog value is completely impractical, but a good example to explain the problems the honest solutions need to address.
First of all there is a limited number of GPIO or ADC-pins on both boards.
On the autopilot board those pins are quite rare, especially because they cannot be shared easily between components.
The second problem is that we do not only need to read a range value from the sensor but we also need to tell the sensor which value to fetch.
Therefore some kind of bidirectional communication between autopilot and sensor need to take place.
The big advantage of using a GPIO pin would be that only one single wire\footnote{Plus two wires for voltage supply} would be needed to connect autopilot and sensor.

\subsection{UART}
The "Universal Asynchronous Receiver/Transmitter"-Protocol uses two wires to establish communication between devices.
\cite{wingen_automatic_2004}

The Disadvantage of UART-style Protocols is that it is a bidirectional connection.
That two pins are needed on sender and reciever side and if another device should be connected two new pins are needed.
On the Lisa/MX autopilot there are 4 dedicated UART connections that might be used, but already 3 of them are used.

\subsection{SPI}

\subsection{CAN bus}

\subsection{I2C}
I2C is a two wired bus protocol that can be used to connect multiple slave devices to one master device.
There already are multiple sensors connected to the autopilot via I2C.
All of the ultrasound-sensors and also the optionalcolorsensor use I2C to communicate with the autopilot.
The autopilot board also supports to have two independent i2c-networks.

Especially the fact that there already is a sensor network on the Finken makes it the best choice as a communication protocol for the new sensor.

\todo{problem: slaves can block the bus}

\section{Communication Protocol}

\todo{polling vs writing to the bus, when data is available}

\todo{"TWI", "Phillips-I2C", "..."}


