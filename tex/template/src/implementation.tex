
\section{Hardware}

\subsection{Ranging Hardware}

\subsection{Autopilot}

\section{Interconnect}
There are diffrent solutions to connecting the ranging board to the Paparazzi autopilot.

\subsection{Pulse width modulation / Analog value}
Using a single GPIO pin or analog value is completely impractical, but a good example to explain the problems the honest solutions need to address.
First of all there is a limited number of GPIO or ADC-pins on both boards.
On the autopilot board those pins are quite rare, especially because they cannot be shared easily between components.
The second problem is that we do not only need to read a range value from the sensor but we also need to tell the sensor which value to fetch.
Therefore some kind of bidirectional communication between autopilot and sensor need to take place.
The big advantage of using a GPIO pin would be that only one single wire\footnote{Plus two wires for voltage supply} would be needed to connect autopilot and sensor.

\subsection{UART}

\subsection{SPI}

\subsection{CAN bus}

\subsection{I2C}

\section{Communication Protocol}

\todo{"TWI", "Phillips-I2C", "..."}


