
\section{Hardware}

\subsection{Ranging Hardware}

There are several different possible hardware plattforms for the Atmel ranging software.
At the time this thesis started using the firmware of the Atmel Ranging Toolbox~\cite{atmelrtb} for the REB233SMAD Evaluation Kit~\cite{REB233SMAD} was the only setup that was already supporting ranging.

For evaluation those modules are quite useable, but there are better options available for use in the real application, as the sensors from the evaluatio kit are quite big and heavy.
It is planned to use 802.15.4 modules from dresden elektronik which are already integrated into the new hardware revision of the FINken robots as telemetrie transmitter.
\todo{other frequency}
Another way to integrate this approach to ranging into the FINken robots is to miniaturize the REB233SMAD-modules, by leaving unused PCB-area and connectors.

\subsection{Assembly}

As it is unclear which version of the ranging hardware is best for the FINken robots there are several ways to fasten the ranging modules.

If a module from dresden elektronik is used it can simply be plugged into the DE-header currently used for the telemerie module.
If a new PCB is used it can be mounted using the fixed hole spacing of \SI{30.5}{mm} that is used by a lot of quatcopter hardware components.

Maybe an extra mounting method is needed for the antenna.
The antenna might be placed on top of the sensor tower to lower interference with the other components of the copter.

\subsection{Autopilot Software}
The FINken robots are controlled by a micro controller handling all the computation needed.
There is no distinction between higher level logic like pathplanning and stabilization of the copter.
The exact board that is used is the LISA/MX autopilot board in hardware revision 2.1~\cite{lisamx} wich runs the paparazzi autopilot firmware\cite{paparazzi}.
The paparazzi framework provides an easy way to implement new hardware drivers for alle devices that are connected to the board via several possible interfaces.

An important design desicion is how to devide the process of ranging and filtering the results between autopilot and sensor node.
The sensor could yield it's raw values to the master device and be done.
This allows to implement fine tuned and application specific filtering, however it also means that there is no convenient way of getting reasonable values that are already filtred.
Another factor that influences this decision is the processing power of the sensor nodes.
Running filters on the sensor node might free up valuable processing power and memory in the master device.

As the FINken robots autopilot is quite well resourced and the measurements are probably not good enough to be used without sophisticated filtering only the raw values of the sensors are used.
However it might still be a good idea to provide higher level data computed by the sensor nodes later on.
Position estimation for example is an application, that needs a lot of computation and memory that is quite simmilar for different applications of the ranging nodes.
\todo{Integration of Position estimate into kalman}

\section{Interconnect}
There are some different methods of connecting the new sensor to the paparazzi autopilot.
All of the following ways of connecting hardware are supported by the LISA/MX board and one has to be chosen for the new ranging sensors.

\subsection{Pulse width modulation / Analog value}
Using a single GPIO pin or analog value is completely impractical, but a good example to explain the problems the honest solutions need to address.

First of all there is a limited number of GPIO or ADC-pins on both boards.
On the autopilot board those pins are quite rare, especially because they cannot be shared easily between components.
The second problem is that we do not only need to read a range value from the sensor but we also need to tell the sensor which value to fetch.
Therefore some kind of bidirectional communication between autopilot and sensor need to take place.
The big advantage of using a GPIO pin would be that only one single wire\footnote{Plus two wires for voltage supply} would be needed to connect autopilot and sensor.

\subsection{UART}
The "Universal Asynchronous Receiver/Transmitter"-Protocol uses two wires to establish communication between devices.
\cite{wingen_automatic_2004}

The Disadvantage of UART-style Protocols is that it is a bidirectional connection.
That two pins are needed on sender and reciever side and if another device should be connected two new pins are needed.
On the Lisa/MX autopilot there are four dedicated UART connections that might be used, but already three of them are used.

\subsection{SPI}
6 Pins, Clocked, Chainable, DMA

\subsection{CAN bus}
Would be ok, no can device yet, pins for canbus already in use, would be solvable, but not easy to implement on ranging nodes

\subsection{I2C}

I2C is a two wired bus protocol that can be used to connect multiple slave devices to one master device.
There already are multiple sensors connected to the autopilot via I2C.
All of the ultrasound-sensors and also the optionalcolorsensor use I2C to communicate with the autopilot.
The autopilot board also supports to have two independent i2c-networks.

Especially the fact that there already is a sensor network on the FINken makes it the best choice as a communication protocol for the new sensor.

\todo{problem: slaves can block the bus}

\todo{Welche funktion für master, welche für slave.}
