
\section{Robustness of Implementation}
\subsection{Bus hangup}

\subsection{i2c errors per time}

\subsection{Integration Test for Quadcopter}


\section{Ranging Accuracy}
\subsection{Antenna Orientation}
\todo{ (diversity, no diversity) * (parallel, perpendicular) }

\subsection{Influence of Distance}

\subsection{Orientation of Devices}


\subsection{Moving Nodes}

\subsection{Ranging on FINken Robots}

\section{Properties of a Distance Function}
The ranging sensor on the FINken robot shall be used to provide a distance between two quadcopters, similar to a distance measure used in swarm intelligence algorithms.
In the pure mathematical sense a distance function has to fullfill certian properties.

If $f(x, y)$ is a distance function it has to have the following properties.
\begin{eqnarray}
f(x, y) \ge 0 \\
f(x, y) = 0 \iff x = y \\ 
f(x, y) = f(y, x) \\ 
f(x, z) \le f(x, y) + f(y, z)
\end{eqnarray}

Of course the value measured by any real sensor will not completly accomplish to satisfy those conditions.
For use in swarm robotics it is therefore very interesting to know in which way the range values break the pattern of a mathematical distance function.

\subsection{Non-negativity and Coincidence}

\subsection{Symmetry}

\subsection{Triangle Inequality}



\section{Conclusion}
