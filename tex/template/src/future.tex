\section{Formation Flying using Swarm Behaviour}
\todo{get simulation data}
One of the next steps is to \todo{übertragen} algorithms from swarm intelligence to the FINken robots.
The robots should be able to form a stable formation by using virtual attraction and repulsion forces to hold a given distance to their neighbors.
If those distances are stable enough formations like triangle meshes can be formed.

\subsection{Direction}
A value that the ranging sensors don't yield is the direction of the other sensor it is ranging with.
For use in swarm algorithms this is a problem: Normally we compute a force towards or from the other swarm entities that is based on distance and direction.
This way is still blocked for us as we can't measure direction.


\section{Flying with Pseudo-GPS}
One of the things that can be done with ranging is position estimation via multilateration.
Even if the FINken project is mainly interested in gaining a viable distance measure between individuals in a swarm a position estimate would be beneficial for the performance of the autopilot-especially as the normal usecase for the Paparazzi autopilot is outdoors and with a GPS reciever attatched.

To integrate positional data into the FINken two steps are needed: Implementing the multilateration algorithm on the sensors and writing a new GPS module that uses the data from the sensor.
An aditional benefit of using anchor nodes to compute a position estimate is that we can find out our current heading and the direction of other swarm entities much easier
