\section{Misc}
\todo{find better caption}

\subsection{Further Integration}

Albeit the range measurements are not great the ranging sensors might still be an improvement to having no ranging capabilities at all.
To achieve this one of the possible hardware solutions from \autoref{ranginghardware} needs to be purchased and integrated into the FINken.
\todo{integration testing, optitrack}

\subsection{Improve Range Values}

Of course the range values can also be improved mathematically.
In this work no filtering of valid measurements was performed.
However the results might be improved by computing range values based on multiple measurements.
A good way of dooing this would be to find a “clever” way of filtering, that takes into consideration the distribution of the values.
To exploit the distribution of the values a better theoretical model for this distribution would be needed.

\todo{formulieren als vorschlag}One way to filter the values like this is by using intervall arithmetics.
As there are big changes in the measured range values when small changes in the distance of the nodes occur the intersection of the intervall might provide a good measure.

\todo{filtering -> papers}



\subsection{Flying with Pseudo-GPS}
One of the things that can be done with ranging is position estimation via multilateration.
Even if the FINken project is mainly interested in gaining a viable distance measure between individuals in a swarm a position estimate would be beneficial for the performance of the autopilot-especially as the normal use case for the Paparazzi autopilot is outdoors and with a GPS reciever attatched.

To integrate positional data into the FINken two steps are needed: Implementing the multilateration algorithm on the sensors and writing a new GPS module that uses the data from the sensor.
An aditional benefit of using anchor nodes to compute a position estimate is that we can find out our current heading and the direction of other swarm entities much easier

\subsection{Distance Based Bounding Box}
\label{boundingbox}
Currently nets and ultrasound reflecting foil are used to enclose the flight area.
Those could be replaced by ranging beacons that enclose the operating area, either by computing a position and defining coordinates which should not be left or by placing ranging nodes in the area and defining a minmum distance to the nearest node.
This could be a nice setup for mobile deployment of the FINken robots.


\section{Applications in Swarm Robotics}

\subsection{Direction}
A value that the ranging sensors don't yield is the direction of the other sensor it is ranging with.
For use in swarm algorithms this is a problem: Normally we compute a force towards or from the other swarm entities that is based on distance and direction.
This way is still blocked for us as we can't measure direction.

\subsection{Distance Based Swarm Objectives}
Swarm behavior can be used for multi objective optimization.
One of those objectives might be based on the measured distance i.e. stay close to a specific node or stay away from a specific node.
Keeping a minimal distance and maximizing the distance between the robots might be used to avoid collisions between multiple robots.
Avoiding collisions of course is one of the most important requirement for ermergence of swarm behavior in a robotic swarm.

\subsection{Collision Avoidance}
Similar to the bounding boxes from \autoref{boundingbox} the distance sensors may be used to enhance collision avoidance in between the copters belonging to the swarm.
This might especially usefull if the safty distance in between the robots shall be higher then the safty distance kept to nearyby objects that are not part of the swarm.
As the copters influence each other by creating a lot of turbulence this strategy might provide benefits for the behavior of the swarm in small environments.

\subsection{Formation Flying}
\todo{get simulation data}
One of the next steps is to \todo{übertragen} algorithms from swarm intelligence to the FINken robots.
The robots should be able to form a stable formation by using virtual attraction and repulsion forces to hold a given distance to their neighbors.
If those distances are stable enough formations like triangle meshes can be formed.

